\documentclass[]{report}
\usepackage{geometry}
\usepackage[utf8]{inputenc}
\usepackage{amsmath}
\usepackage{amsfonts}
\usepackage{amssymb}
\usepackage[pdftex]{graphicx}
\usepackage[space]{grffile}
\usepackage{epstopdf}
\usepackage{epsfig}
\usepackage{bmpsize}
\graphicspath{{./images/}}
\usepackage[export]{adjustbox}
\usepackage{wrapfig}
\usepackage{appendix}
\usepackage{listings}
\usepackage{hyperref}
\usepackage{float}
\hypersetup{
	colorlinks=true,
	allcolors=blue,
	pdfborderstyle={/S/U/W 1}
}
\usepackage{hypcap}
\usepackage{subcaption}

% Add semantic reference commands
\newcommand*{\figref}[1]{%
	\hyperref[{#1}]{Figure~\ref*{#1}}%
}%
\newcommand*{\chapref}[1]{%
	\hyperref[{#1}]{\nameref{#1},~Chapter~\ref*{#1}}%
}%
\newcommand*{\secref}[1]{%
	\hyperref[{#1}]{\nameref{#1},~Section~\ref*{#1}}%
}%

\begin{document}

\title{LSF Modulhandbuchpflege}
\author{James Antrim}

\begin{figure}
\centering
\includegraphics[width=7cm]{THM_Logo_200px.png}
\maketitle
\end{figure}

\newpage

\chapter{LSF Modulhandbuchpflege}

Dieses Dokument gibt einen Muster nach dem die Pflege der Modulhandbücher in LSF erfolgen könnte.

\section{Modultitel}

Die Modultitel ist der Name des Moduls. Dies ist selbst erklärend

\section{Modulcode (alphanumerisch)}

Eine Folge von Zeichen die das Modul kennzeichnen. Zur Zeit wird dieser Feld unterschiedlich und uneindeutig von allen Fachbereichen gepflegt.


\end{document}